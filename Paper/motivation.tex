\section{Motivation}
\label{sec:motivation}

% add reference to google fs paper %
With the widespread use of commodity hardware in many systems such as
the Google File System, error recovery has become the new focus. In
environments such as these errors and failures are accepted as the
norm and error recovery techniques become the critical components of
the system. For a system employing block level caching failures in the
system would mean fetching blocks from network storage all over again
and attempting to get the system back to its previous state so it can
continue its task. We wish to mitigate this issue by allowing easy
recovery of a cache in the event of a failure this way nodes that have
failed can continue their tasks from where they left off with minimal
recovery overhead.

Recovering the data that was already present in the cache would
mitigate recovery time since the cache will already be warm which
means that blocks that the system was caching before the failure would
once again be available rather than having to be fetched from remote
storage again. Fault tolerance comes at a cost though, because there
needs to be a way to store data persistently somewhere this mechanism
would add overhead but with emerging SSD technology and the response
time and throughput of SSD's the overhead that would be incurred under
normal operation is acceptable in some cases if it guarantees quick
recovery from a failure.
