\section{Introduction}

We have designed and implemented fault tolerance mechanisms in the
Linux kernel module DM-Cache~\cite{DM-Cache}. With little overhead
this mechanism allows for cache recovery in the event of a system
failure. Our scheme alleviates the network traffic that would normally
happen from having to retrieve blocks from the storage network that
were already present on the cache. We can also recover blocks that
would have otherwise been lost that had not been written back to the
network which would have resulted in lost data.

The remainder of this paper is organized as follows. In the following
section we provide background information on block level caching
mechanisms present in the Linux kernel. We discuss the motivation for
this work in Section~\ref{sec:motivation}. Section~\ref{sec:related}
discusses related work and current approaches to fault tolerance in
enterprise level systems. Section~\ref{sec:implementation} details the
design and implementation of our fault tolerance mechanism in
DM-Cache. In Section~\ref{sec:evaluation} we present our test setup
and evaluation of our implemented mechanism with various I/O
benchmarks as well as overhead costs. Our discussion and future work
is detailed in Section~\ref{sec:discussion} and
Section~\ref{sec:future} respectively and we conclude in
Section~\ref{sec:conclusion}.
