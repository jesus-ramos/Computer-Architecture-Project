\section{Design and Implementation}
\label{sec:implementation}

\subsection{Design Considerations}

The two main design considerations of our implementation are data
consistency and overhead. With data consistency we want to make sure
that we never have invalid data in the cache's metadata that could
cause corruption. We also want to make sure that we have an acceptable
overhead so as to not impact the performance of the cache.

\subsubsection{Data Consistency}

\subsubsection{Overhead}

\subsection{Implementation}

We leverage current mechanisms implemented in DM-Cache for metadata
management. The current implementation employs a persistence scheme
that allows you to write all the metadata for the cache upon
destruction of the cache. The problem with this implementation is that
in the event of a system failure we don't have the convenience of this
code being executed so all metadata is lost.

To overcome these limitations we came up with a new way to represent
metadata on the cache device as well as a persistence mechanism that
ensures data consistency.

\subsubsection{Metadata Representation}

The metadata representation in the current version of DM-Cache only
takes into account valid blocks or blocks that do not have pending
changes in the cache. We extend this by allowing it to also handle
blocks that have been marked as dirty and are pending to be written
back to the storage device. We do this by creating a small structure
which contains the sector number and the state of that block on the
disk and write this to the end of the disk along with a metadata
header for the whole cache. We can use this header to determine how
many blocks of metadata to read from the disk in order to restore
previous cache state.

When the cache is being reconstructed we simply read this header off
the end of the disk to reinitialize the cache parameters and using
this data we also read the metadata and restore the mapping in the
cache to its previous state.

\subsubsection{Persistence Mechanism}

% Journal style transactions, invalidate, write data, write metadata
% (validate) need to get some numbers to check this though %
