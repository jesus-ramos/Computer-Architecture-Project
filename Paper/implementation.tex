\section{Design and Implementation}
\label{sec:implementation}

\subsection{Design Considerations}

\subsection{Implementation}

We leverage current mechanisms implemented in DM-Cache for metadata
management. The current implementation employs a persistence scheme
that allows you to write all the metadata for the cache upon
destruction of the cache. The problem with this implementation is that
in the event of a system failure we don't have the convenience of this
code being executed so all metadata is lost.

\subsubsection{Metadata Representation}

The metadata representation also only took into account valid blocks
in the cache. We extend this by allowing it to also handle blocks that
have been marked as dirty and are pending to be written back to the
storage device. We do this by creating a small structure which
contains the block number and the state of that block on the disk and
write this to the end of the disk along with a metadata header for the
whole cache. We can use this header to determine how many blocks of
metadata to read from the disk in order to restore previous cache
state.

\subsubsection{Persistence Mechanism}

% Journal style transactions, invalidate, write data, write metadata
% (validate) need to get some numbers to check this though %
