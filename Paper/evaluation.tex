\section{Evaluation}
\label{sec:evaluation}

We have implemented metadata persistence on the latest version of
DM-Cache running on Linux kernel version 3.3. To evaluate this we use
the workloads offered by the IOZone benchmark to evaluate the overhead
of this mechanism.

We use two testing machines and iSCSI and assign one of the machines
to be the client and the other the storage target using Open-iSCSI. We
evaluated our approach on an Mtron Pro 7500 64GB SSD. We used an Mtron
SSD to get a good upper bound on our performance overhead as Mtron
SSD's (as compared to Intel and Vertex SSD's) have slower write
performance \cite{FIOS}.

Due to issues with the implementation of the write-back mechanism
within DM-Cache we only show evaluation of DM-Cache using our
persistence mechanism with write-through. Our testing shows no issues
with the actual data when running experiments but the kernel logs show
some file-system warnings so we omit those results.

\subsection{IOZone}

We use the IOZone benchmark to evaluate the throughput of DM-Cache as
a cold cache with and without metadata persistence enabled as well as
a warm cache with and without metadata persistence enabled to
calculate the overhead.

Because we are only using the write-through mechanism in our
evaluation the only throughput results that are meaningful to us are
the read throughput statistics. Reread is not significant because that
will use the data that is cached in RAM.

For each test run we create and initialize a new 2GB cache and run
IOZone with different I/O sizes and record the results. To evaluate
the results of a warm cache we simply run the read workload in IOZone
and then drop the page cache for the system so that it is forced to do
I/O rather than simply reading the information from RAM while the
information is still stored in the cache.

% put some figures with the numbers in %
