\section{Related Work}
\label{sec:related}

The argument for SSD's in enterprise solutions was proposed by Lee et
al.\ where they were able to show that SSD's could improve the
throughput of enterprise database applications
\cite{EnterpriseSSD}. DM-Cache was created for extending this
functionality to network storage devices for throughput improvement of
remote storage. Our work builds upon DM-Cache by providing a fault
tolerant persistence layer to alleviate the costs of system failure
once the system is brought back online.

Previous work has been done before on communication architectures to
alleviate network bandwidth issues such as work by Al-Fares et al.\
which involved using hierarchies of commodity switches
\cite{Network}. With this work we seek to eliminate or reduce the
effect of failures on network storage by implementing mechanisms to
recover those recently used blocks on the cache. This means that we
can alleviate data requests across the network to retrieve the lost
cached blocks.

Error recovery in caching systems is not a new idea and has been shown
to work effectively in cloud based systems such as RAMCloud
\cite{RAMCloud}. With this work we extend that idea by providing fault
tolerance capabilities to block level caching mechanisms present in
the Linux kernel.

Work done by Saxena et al.\ on FlashTier is very similar to this work
in that it provides cache consistency allowing the cached data to be
used after a crash or failure \cite{flashtier}. Their work is mainly
done in the hardware interaction between the SSD and the operating
system allowing for a more powerful and efficient interface when it
comes to caching solutions that use SSD's. Our work focuses more on an
existing implementation that is in use and how to add functionality to
that on the software end.

Efficiently handling metadata on SSD's is difficult due to the
obscurity of what happens in the underlying hardware, this is
especially difficult for caching mechanisms where many improvements
can be made. A lot of work has gone into finding more efficient ways
of representing and handling metadata on disk such as work on the
FD-Tree structure and the Flash B-Tree \cite{FlashB-Tree,
  FD-Tree}. Our work only uses a naive implementation based on the
hash table structure in DM-Cache used to represent the block mappings
\cite{DM-Cache}. We do this for simplicity as it requires less changes
to the current code and our work is more of a proof of concept.
