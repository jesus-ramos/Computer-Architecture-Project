\section{Related Work}
\label{sec:related}

The argument for SSD's in enterprise solutions was proposed by Lee et
al.\ where they were able to show that SSD's could improve the
throughput of enterprise database
applications~\cite{EnterpriseSSD}. DM-Cache was created for extending
this functionality to network storage devices for throughput
improvement of remote storage. Our work builds upon DM-Cache by
providing a fault tolerant persistence layer to alleviate the costs of
system failure once the system is brought back online.

Previous work has been done before on communication architectures to
alleviate network bandwidth issues such as work by Al-Fares et al.\
which involved using hierarchies of commodity
switches~\cite{Network}. With this work we seek to eliminate or reduce
the effect of failures on network storage by implementing mechanisms
to recover those recently used blocks on the cache. This means that we
can alleviate data requests across the network to retrieve the lost
cached blocks.

Error recovery in caching systems is not a new idea and has been shown
to work effectively in cloud based systems such as
RAMCloud~\cite{RAMCloud}. With this work we extend that idea by
providing fault tolerance capabilities to block level caching
mechanisms present in the Linux kernel.
