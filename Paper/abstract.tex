\begin{abstract}

  System failure has become the norm in commodity hardware based
  systems. Rather than try to avoid errors completely they accept that
  errors will happen and instead find ways to more quickly
  recover. For systems that employ block level caching and remote
  storage networks this usually results in having to fetch recently
  used blocks from the storage area network once again to warm up the
  cache once more. To address this problem we have implemented a
  metadata persistence mechanism to DM-Cache to allow recovery of the
  cache in the event of a system failure.

  Our metadata persistence scheme is implemented and evaluated using
  the latest version of DM-Cache on Linux kernel version 3.3. It's
  overhead was evaluated using the IOZone benchmark on a network
  storage drive using Open-iSCSI. The results show that the overhead
  costs can outweigh the benefits for read heavy workloads and cache
  recovery time is almost instant. Our experiments also showed no data
  corruption in the event of simulated failures and that recovering
  the cache from metadata allowed the throughput of the system to
  resume what it was before the failure without having to retrieve all
  the cached blocks from network storage again.

\end{abstract}
