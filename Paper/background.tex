\section{Background}

\subsection{Storage Area Networks}

\subsection{Enterprise SSD Usage}

\subsection{DM-Cache}

DM-Cache is a block-level caching mechanism implemented as a module in
the Linux kernel~\cite{DM-Cache}. It allows for specification of a
cache device where recently used blocks retrieved from network storage
can be stored for faster access time. It supports both write-through
and write-back functionality for handling cached data.

The current metadata persistence scheme in DM-Cache only works if the
cache is properly destroyed. In the event of a failure no metadata is
persisted to disk and the metadata that contains the location of the
cached blocks on the disk is lost.
