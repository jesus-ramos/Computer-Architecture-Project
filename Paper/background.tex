\section{Background}

\subsection{Storage Area Networks}

\subsection{Enterprise SSD Usage}

\subsection{DM-Cache}

DM-Cache is a block-level caching mechanism implemented as a module in
the Linux kernel~\cite{DM-Cache}. It allows for specification of a
cache device where recently used blocks retrieved from network storage
can be stored for faster access time. It supports both write-through
and write-back functionality for handling cached data.

The purpose of DM-Cache is to cache recently used blocks on local
storage instead of retrieving them from some external storage source
every time they are needed. This helps alleviate network bandwidth as
well as reduce latency and improve response time to requests.

The current metadata persistence scheme in DM-Cache only works if the
cache is properly destroyed. In the event of a failure no metadata is
persisted to disk and the metadata that contains the location of the
cached blocks on the disk is lost.

Another limitation of this scheme is that only blocks that are valid
or have not been modified have their metadata persisted. When a cache
is destroyed all dirty blocks have their changes written back to the
storage device (in the case of a write-back cache) but in the event of
a failure this is not true as blocks can still be pending write back
during this time.
