\documentclass[letterpaper,12pt]{article}

\usepackage[margin=1in]{geometry}
\usepackage{titling}

\setlength{\droptitle}{-0.75in}

\title{Fault Tolerance in Block-Level Caching}
\author{
  Jesus Ramos \\ \texttt{jramo028@fiu.edu} \and
  Douglas Otstott \\ \texttt{dotst001@fiu.edu}
}
\date{}

\pagestyle{empty}

\begin{document}

\maketitle
\thispagestyle{empty}

\section*{Problem Statement}

Dependability and reliability are two important factors when dealing
with cloud computing. Systems should be robust and be able to recover
from errors and failure effectively and quickly. \\ \\*
%
Fault tolerance is still a major issue in cloud based systems that
employ caching. Currently in the event of an error or power failure
all data that was pending to be written back is lost along with
currently cached data. In some cases minor data loss is acceptable,
but in most cases even minor loss of data can result in larger
problems and error recovery can become very complex depending on the
type and quantity of data lost.

% end section Problem Statement

\section*{Problem Background}

Many cloud based systems employ network storage due to the physical
limitations of local storage devices. While this solves the storage
size problem it also increases the response latency and might have
bandwidth issues due to the sheer amount of requests going to the
network storage devices. \\ \\*
%
Caches are employed by many systems as a means to bring in the most
frequently used data to more local storage devices to reduce latency
and alleviate pressure on the main storage system.

% end section Problem Background

\section*{Proposed Solution}

Our proposed solution is to add a layer of fault tolerance to DM-Cache
to allow for more graceful recovery of the errors mentioned
previously. \\ \\*
%
The main goal of this project is to allow recovery of a cache that was
lost due to a power failure or possible system crash. This way, once
the machine that was running the cache is brought back online, it is
able to recover the information that it was caching. This solves the
issue of lost write back data and also allows the machine to continue
from where it left off by having a warm cache. This further alleviates
pressure on the main storage system by allowing graceful recovery of
data that was already present in the cache, meaning that applications
can start running much faster than in situations where there was no
fault tolerance.

% end section Proposed Solution

\section*{Project Timeline}

The first phase of this project is to do some research on methods for
persisting cache meta-data effectively so that recovery can be done
quickly and efficiently when the cache is being recreated. \\ \\*
%
Once we have decided on the method for persisting the data we will
actually implement this into DM-Cache and add an option when creating
the cache to allow for recovery of the previous cache in case of
system failure. \\ \\*
%
The last phase will involve testing and evaluation of just how robust
the system will be in the even of a system failure such as power or
system crash.

% end section Project Timeline

\section*{Project Evaluation}

% end section Project Evaluation

\end{document}
